\documentclass[french]{article}
\usepackage[T1]{fontenc}
\usepackage[french]{babel}
\usepackage[utf8]{inputenc}
%\usepackage{a4wide}
\usepackage{amssymb}
\usepackage{amsmath}

% --JUGEMENTS-- %
\newcommand{\jugementPointeur}{
        \dfrac{\sigma \vdash id : \tau}
              {\sigma \vdash *id : \tau, \pi} \text{où }\pi \text{ est la marque.}\\
        \text{Par suite, } \tau_r \text{ est l'environnement de type et de marque.}\\
        \text{Les jugements de typages sont alors inchangés, simplement on peut remplacer}\\
        \text{pour les variables id par *id en suivant le typage ci-dessus.}
        }

\newcommand{\jugementElseOpt}{
        \dfrac{\sigma \vdash E : \text{bool} \hspace*{10pt} \sigma, \tau_r \vdash \text{BLOC} : \text{void}}
              {\sigma, \tau_r \vdash \text{if } E \text{ BLOC} : \text{void}, []}
        }

\newcommand{\jugementTernaire}{
        \dfrac{\sigma \vdash E : \text{bool} \hspace*{10pt} \sigma \vdash E_1 : \tau \hspace*{10pt} \sigma \vdash E_2 : \tau }
              {\sigma \vdash (E \text{ ? } E_1 : E_2) : \tau}
        }

\newcommand{\jugementLoop}{
        \dfrac{\sigma, \tau_r \vdash \text{BLOC} : \text{void}}
              {\sigma, \tau_r \vdash \text{loop} \text{ BLOC} : \text{void}, []}
        }
\newcommand{\jugementLoopId}{
        \dfrac{id::\sigma, \tau_r \vdash \text{BLOC} : \text{void}}
              {\sigma, \tau_r \vdash \text{define id : loop} \text{ BLOC} : \text{void}, []}
        }

\newcommand{\jugementIncrementPost}{
        \dfrac{\sigma \vdash id : \text{Int}}
              {\sigma \vdash \text{id}++ : \text{Int}}
        }

\newcommand{\jugementIncrementPre}{
        \dfrac{\sigma \vdash id : \text{Int}}
              {\sigma \vdash ++\text{id} : \text{Int}}
        }

\begin{document}


\title{\textbf{Traduction des Langages}}
\author{Quentin \textsc{Fraty}\\
        Nathan \textsc{Maillet}}
\date{}

\maketitle

% Transversale (?)
% !! Penser a traiter les évolutions des AST !!
% Justifications pertinentes et complètes sur l'evolution de la structure des AST.
% Comparaison avec d'autres choix de conception
% Effacer les lignes de ce message UNIQUEMENT lorsque qu'elles ont été traitées

\section{Introduction}
% Bonne description du sujet et des points abordés dans la suite du rapport
% pas de copier/coller du sujet !!
% Effacer les lignes de ce message UNIQUEMENT lorsque qu'elles ont été traitées

\section{Mutation de \emph{tds.ml} en \emph{mtds}}
% Rq : mtds a pour vocation de généraliser tds
%      plutôt que de changer tds.ml pour inclure les ptrs et
%      galérer dans le cas où on veut revenir en arrière ou modifier
%      les identifiants a l'avenir (casts, expliciter les kinds -btw, ça existe ? ça marche comment en cpp ?)
%      il n'y aura pas besoin de changer mtds
%      -> cela nous a été utile car on a changé de stratégie pour les ptrs
%      mtds n'est pas contraint par une structure de monade car je ne vois pas l'intérêt dans le monde OCaml pré version 5...
% Effacer les lignes de ce message UNIQUEMENT lorsque qu'elles ont été traitées

\section{Pointeurs}
% Explications pertinentes et complètes sur leur traitement (sans...). Comparaison avec d’autres choix de conception.
% Ne pas oublier les jugements de typage !
% Rq : question cruciale : est-ce que int *a c'est ( int * ) a ou int ( *a ) ? 
%         choix : int ( *a ), on traite ( *a )
%         comme l'indentifiant, constitué d'un marqueur : * et d'un symbole/identifiant : a 
% Effacer les lignes de ce message UNIQUEMENT lorsque qu'elles ont été traitées

\section{Bloc else optionnel}
% Explications pertinentes et complètes sur leur traitement (sans...). Comparaison avec d’autres choix de conception.
% Ne pas oublier les jugements de typage !
% Effacer les lignes de ce message UNIQUEMENT lorsque qu'elles ont été traitées

\section{Conditionnelle ternaire}
% Explications pertinentes et complètes sur leur traitement (sans...). Comparaison avec d’autres choix de conception.
% Ne pas oublier les jugements de typage !
% Effacer les lignes de ce message UNIQUEMENT lorsque qu'elles ont été traitées

\section{Loop à la Rust}
% Explications pertinentes et complètes sur leur traitement (sans...). Comparaison avec d’autres choix de conception.
% Ne pas oublier les jugements de typage !
% Rq : Problème de la pile a évoqué
% Effacer les lignes de ce message UNIQUEMENT lorsque qu'elles ont été traitées

\section{Surcharge de fonctions}
% Explications pertinentes et complètes sur leur traitement (sans...). Comparaison avec d’autres choix de conception.
% Effacer les lignes de ce message UNIQUEMENT lorsque qu'elles ont été traitées

\section{Incréments}
% Explications pertinentes et complètes sur leur traitement (sans...). Comparaison avec d’autres choix de conception.
% Ne pas oublier les jugements de typage !
% Effacer les lignes de ce message UNIQUEMENT lorsque qu'elles ont été traitées

\section{Affichage des erreurs}
% Explications pertinentes et complètes sur leur traitement (sans...). Comparaison avec d’autres choix de conception.
% Effacer les lignes de ce message UNIQUEMENT lorsque qu'elles ont été traitées

\section{Conclusion}
% Bon recul sur les difficultés rencontrées et améliorations éventuelles
% Améliorations éventuelles :
% On n'autorise pas les fonctions d'ordre supérieur avec les pointeurs mais
% attention a gérer le LB !!
% On autorise les pointeurs dans les paramètres mais pas en retour
% Effacer les lignes de ce message UNIQUEMENT lorsque qu'elles ont été traitées

\end{document}
